% !TEX root = ../main.tex
\subsection{Axioms of Probability}
\begin{boks}{Definition 1.3 (Axioms of Probability)}
  A \textit{probability measure} is a function $P$, which assigns to each event $A$ a number of $P(A)$ satisfying
  \begin{enumerate}[label = \textbf{(\alph*)}]
    \item $0 \leq P(A) \leq 1$
    \item P(S) = 1
    \item If $A_1, A_2, \ldots$ is a sequence of \textit{pairwise disjoint} events, that is, if $i \neq j$, then $A_i \cap A_j = \emptyset$, then
    \begin{align*}
      P\left( \bigcup_{k = 1}^\infty A_k \right) = \sum_{k = 1}^\infty P(A_k)
    \end{align*}
  \end{enumerate}
\end{boks}
\subsection{Conditional Probability}
\begin{boks}{Definition 1.4}
  Let $B$ be an event such that $P(B) > 0$. For any event $A$, denote and define the \textit{conditional probability} of $A$ \textit{given} $B$ as
  \begin{align*}
    P(A|B) &= \frac{P(A \cap B)}{P(B)}, \quad \text{or}\\
    P(A|B)P(B) &= P(A \cap B)
  \end{align*}
\end{boks}

\subsection{Law of Total Probability}
\begin{boks}{Theorem 1.1 (Law of Total Probability)}
    Let $B_1, B_2, \ldots$ be a sequence of events such that
    \begin{enumerate}
        \item $P(B_k) > 0$ for $k = 1, 2, \ldots$
        \item $B_i$ og $B_j$ are disjoint whenever $i \neq j$
        \item $S = \bigcup_{k=1}^\infty B_k$
    \end{enumerate}
    Then, for any event $A$, we have
    \begin{align*}
        P(A) = \sum_{k=1}^\infty P(A|B_k)P(B_k).
    \end{align*}
\end{boks}

\begin{proof}
First note that
\begin{align*}
    A = A \cap S = \bigcup_{k = 1}^\infty (A \cap B_k),
\end{align*}
by the distributive law for infinite unions. Since $A \cap B_1, A \cap B_2, \ldots$ are pairwise disjoint,  we get
\begin{align*}
    P(A) = P \left(\bigcup_{k = 1}^\infty (A \cap B_k)\right) = \sum_{k = 1}^\infty P(A \cap B_k) = \sum_{k = 1}^\infty P(A|B_k)P(B_k).
\end{align*}
Which proves the theorem. Note that the result also holds for finite sequences.
\end{proof}

\begin{boks}{Corollary 1.6}
If $0 < P(B) < 1$, for $B\subseteq S$, then
\begin{align*}
    P(A) = P(A|B)P(B) + P(A|B^c)P(B^c)
\end{align*}
\end{boks}

\subsection{Bayes Formula}
\begin{boks}{Proposition 1.11 (Bayes' Formula)}
Under the same assumptions as in the law of total probability and if $P(A) > 0$, then for any event $B_j$, we have
\begin{align*}
    P(B_j|A) &= \frac{P(A|B_j)P(B_j)}{\sum_{k=1}^\infty P(A|B_k)P(B_k)} \\
    &= \frac{P(A|B_j)P(B_j)}{P(A)} \qquad \text{according to the LTP.}
\end{align*}
\end{boks}
\begin{proof}
  Note that, by the law of total probability, the denominator is nothing but $P(A)$, and hence we must show that
  \begin{align*}
    P(B_j|A) = \frac{P(A|B_j)P(B_j)}{P(A)}
  \end{align*}
  which is to say that
  \begin{align*}
    P(B_j|A)P(A) = P(A|B_j)P(B_j)
  \end{align*}
  which is true since both sides equal $P(A \cap B_j)$, by the definition of conditional probability.
\end{proof}

\begin{boks}{Corollary 1.7}
  If $0 < P(B) < 1$ and $P(A) > 0$, then
  \begin{align*}
    P(B|A) = \frac{P(A|B)P(B)}{P(A|B)P(B) + P(A|B^c)P(B^c)}
  \end{align*}
\end{boks}
