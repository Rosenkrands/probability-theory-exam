% !TEX root = ../main.tex
\subsection{Continuous Stochastic Variable}
\begin{boks}{Definition 2.5}
If the cdf $F$ is a continuous and differentiable function, then $X$ is said to be a \textit{continuous random variable}.
\end{boks}

\subsection{Cumulative Distribution Function}
\begin{boks}{Proposition 2.3}
If $F$ is the cdf of any random variable, $F$ has the following properties:
\begin{enumerate}
    \item It is nondecreasing
    \item It is right-continuous
    \item It has the limits $F(-\infty) = 0$ and $F(\infty) = 1$ (where the limits may or may not be attained at finite $x$).
\end{enumerate}
\end{boks}

% \begin{boks}{Proposition 2.4}
% Let $X$ be any random variable with cdf $F$. Then
% \begin{enumerate}
%     \item $P(a < X \leq b) = F(b) - F(a), \quad a \leq b$
%     \item $P(X > x) = 1 - F(x), \quad x \in \mathbb{R}$
%     \item $F$ is continuous from the right with limit form the left and increasing
% \end{enumerate}
% \end{boks}

\subsection{Probability Density Function}
\begin{boks}{Definition 2.6}
The function $f(x) = F'(x)$ is called the \textit{probability density function} (pdf) of $X$.
\end{boks}
\begin{minipage}{0.7\textwidth}
  \begin{boks}{Proposition 2.5}
  Let X be a continuous random variable with pdf $f$ and cdf $F$. Then
  \vspace{5mm}
  \begin{enumerate}
      \item $F(x) = \int_{-\infty}^x f(t)dt, \quad x \in \mathbb{R}$
      \item $f(x) = F'(x), \quad x \in \mathbb{R}$
      \item For $B \subseteq \mathbb{R}, \quad P(X\in B) = \int_B f(x)dx$
  \end{enumerate}
  \end{boks}
\end{minipage}
\begin{boks}{Proposition 2.6}
A function $f$ is a possible pdf of some continuos random variable if and only if
\begin{enumerate}
    \item $f(x) \geq 0, \quad x \in \mathbb{R}$
    \item $\int_{-\infty}^\infty f(x)dx = 1$
\end{enumerate}
\end{boks}

\subsection{Proposition 2.8}
\begin{boks}{Proposition 2.8}
Let $X$ be a continuous random variable with pdf $X$, let $g$ be a strictly increasing or strictly decreasing, differentiable function, and let $Y = g(X)$. Then $Y$ has pdf
\begin{align*}
    f_Y(y) = \bigg| \frac{d}{dy}g^{-1}(y)\bigg| f_X(g^{-1}(y))
\end{align*}
for $y$ in range of $Y$.
\end{boks}
\begin{proof}
  \begin{align*}
      F_Y(y)&=P(Y\leq y)=P(g(X)\leq y)=P(X\leq g^{-1}(y))=F_X(g^{-1}(y))
  \end{align*}
  \begin{align*}
      f_Y(y)&=F'_Y(y)=F'_X(g^{-1}(y))= \frac{d}{dy}g^{-1}(y)\cdot F'_X(g^{-1}(y))=\frac{d}{dy}g^{-1}(y)\cdot f_X(g^{-1}(y))
  \end{align*}
\end{proof}

\subsection{Expected Value}
\begin{boks}{Defintion 2.9}
Let $X$ be a continuous random variable with pdf $f$. The \textit{expected value} of $X$ is defined as
\begin{align*}
    E[X] = \int_{-\infty}^\infty xf(x)dx = \int_\mathbb{R} xf(x)dx,
\end{align*}
notice that the last equality is not always satisfied, but for the purpose of this course it is.
\end{boks}

% \begin{boks}{Proposition 2.10}
% Let $X$ be a continuous random variable with range $[0, \infty)$. Then
% \begin{align*}
%     E[x] = \int_0^\infty P(X > x)dx
% \end{align*}
% \end{boks}

% \begin{boks}{Proposition 2.11 (Linearity for the Expectation)}
% Let $X$ be any random variable, and let $a$ and $b$ be real numbers. Then
% \begin{align*}
%     E[aX + b] = aE[X] + b
% \end{align*}
% \end{boks}
%
% \begin{proof}
% We prove this in the continuous case, for $a > 0$. Let $Y = aX + b$, and note that $Y$ is a continuous random variable, which by proposition 2.8 has pdf
% \begin{align*}
%     f_Y(y) = \frac{1}{a} f_X\bigg(\frac{y-b}{a}\bigg)
% \end{align*}
% and by definition, the expected value of $Y$ is
% \begin{align*}
%     E[Y] = \int_{-\infty}^\infty yf_Y(y)dy = \frac{1}{a}\int_{-\infty}^\infty yf_X\bigg(\frac{y-b}{a}\bigg)dy
% \end{align*}
% where the variable $y=ax+b$ gives $dy = a \ dx$ and hence
% \begin{align*}
%     E[Y] &= \int_{-\infty}^\infty (ax + b)f_X(x)dx \\
%     &= a \int_{-\infty}^\infty xf_X(x)dx + b \int_{-\infty}^\infty f_X(x)dx \\
%     &= aE[X] + b
% \end{align*}
% and we are done.
% \end{proof}

\subsection{Variance}
\begin{boks}{Definition 2.10}
Let $X$ be a random variable with expected value $\mu$. The \textit{variance} of $X$ is defined as
\begin{align*}
    Var[X] = E\Big[ (X - \mu)^2 \Big]
\end{align*}
\end{boks}

% \begin{boks}{Definition 2.11}
% Let $X$ be a random variable with variance $\sigma^2 = Var[X]$. The \textit{standard deviation} of $X$ is then defined as $\sigma = \sqrt{Var[X]}$.
% \end{boks}

\begin{boks}{Corollary 2.2}
\begin{align*}
    Var[X] = E[X^2] - (E[X])^2
\end{align*}
\end{boks}

\begin{proof}
By proposition 2.12, $E[g(X)] = \int_{-\infty}^\infty g(x)f_X(x)dx$, we have
\begin{align*}
    Var[X] &= E\Big[(X-\mu)^2\Big] = \int_{-\infty}^\infty (x - \mu)^2f(x) dx\\
    &= \int_{-\infty}^\infty (x^2 - 2x\mu + \mu^2) f(x) dx \\
    &= \int_{-\infty}^\infty x^2 f(x) dx - 2\mu \int_{-\infty}^\infty x f(x) dx + \mu^2\int_{-\infty}^\infty f(x)dx\\
    &=E[X^2] - 2\mu E[X] + \mu^2\\
    &= E[X^2] - 2 E[X]^2 + (E[X])^2\\
    &= E[X^2] - (E[X])^2.
\end{align*}
\end{proof}

% \begin{boks}{Proposition 2.14 (Chebyshev's Inequality)}
% Let $X$ be any random variable with mean $\mu$ and variance $\sigma^2$. For any consatn $c > 0$, we have
% \begin{align*}
%     P(|X - \mu| \geq c\sigma) \leq \frac{1}{c^2}.
% \end{align*}
% \end{boks}
%
% \begin{proof}
% Let us prove the continuous case. Fix $c$ and let $B$ be the set $\{x \in \mathbb{R} : |x - \mu| \geq c\sigma\}$. We get
% \begin{align*}
%     \sigma^2 &= E[(X - \mu)^2] = \int_{-\infty}^\infty (x-\mu)^2f(x)dx=\int_\mathbb{R}(x-\mu)^2f(x)dx\\
%     &\geq \int_B(x-\mu)^2f(x)dx \geq c^2\sigma^2\int_Bf(x)dx = c^2\sigma^2P(X\in B).
% \end{align*}
% Which gives the desired inequality.
% \end{proof}
