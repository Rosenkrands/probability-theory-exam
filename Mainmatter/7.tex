% !TEX root = ../main.tex
% When we introduced expected values, we argued that these could be considered averages of a large number of observations. Thus, if we have observations $X_1, X_2, \ldots, X_n$ and we do not know the mean $\mu$, a reasonable approximation ought to be the \textit{sample mean}
% \begin{align*}
%   \bar{X} = \frac{1}{n} \sum_{k = 1}^{n} X_k
% \end{align*}
% in other words, the average of $X_1, \ldots, X_n$. Suppose now that the $X_k$ are i.i.d. with mean $\mu$ and variance $\sigma^2$. By the formulas for the mean and variance of sums of independent variables, we get
% \begin{align*}
%   E[\bar{X}] = E\left[ \frac{1}{n} \sum_{k = 1}^{n} X_k \right] = \sum_{k = 1}^{n} \frac{1}{n} E[X_k] = \mu
% \end{align*}
% and
% \begin{align*}
%   Var[\bar{X}] = Var \left[\frac{1}{n} \sum_{k = 1}^{n} X_k\right] = \sum_{k = 1}^{n} \frac{1}{n^2} Var[X_k] = \frac{\sigma^2}{n}
% \end{align*}
% that is, $\bar{X}$ has the same expected value as each individual $X_k$ and a variance that becomes smaller the larger the value of $n$.

% \subsection{The Law of Large Numbers}
%
% ALthough we can nevet guarantee that $|\bar{X} - \mu|$ is smaller than a given $\varepsilon$ we can say that is very likely that $|\bar{X} - \mu|$ is small if $n$ is large. That is the idea behind the following result.
% \begin{boks}{Theorem 4.1. (The Law of Large Numbers)}
%   Let $X_1, X_2, \ldots$ be a sequence of i.i.d. random variables with mean $\mu$, and let $\bar{X}$ be their sample mean. Then, for every $\varepsilon > 0$
%   \begin{align*}
%     P(|\bar{X} - \mu| > \varepsilon) \rightarrow 0 \quad
%     \text{as} \quad n \rightarrow \infty
%   \end{align*}
% \end{boks}
% \begin{proof}
%   Assume that the $X_k$ have finite variance, $\sigma^2 < \infty$.
%   Apply Chebyshev's inequality to $\bar{X}$ and let $c = \varepsilon \sqrt{n}/\sigma$.
%   Since $E[\bar{X}] = \mu$ and $Var[\bar{X}] = \sigma^2/n$, we get
%   \begin{align*}
%     P(|\bar{X} - \mu| > \varepsilon) \leq \frac{\sigma^2}{n\varepsilon^2} \rightarrow 0 \quad
%     \text{as} \quad n \rightarrow \infty
%   \end{align*}
% The assumptions of finite variance is neccessary for this proof to work. However, the law of large numbers is tru also if the varinace is infinite, but the proof in that case is more involved and we will not give it.
% \end{proof}
% We say that $\bar{X}$ \textit{converges in probability} to $\mu$ and write
% \begin{align*}
%   \bar{X} \xrightarrow{P} \mu \quad \text{as} \quad n \rightarrow \infty
% \end{align*}
% \begin{boks}{Corollary 4.1}
%   Consider an experiment where the event $A$ occurs with probability $p$. Repeat the experiment independently, let $S_n$ be the number of times we get the event $A$ in $n$ trials, and let $f_n = S_n/n$, the relative frequency. Then
%   \begin{align*}
%     f_n \xrightarrow{P} p \quad \text{as} \quad n \rightarrow \infty
%   \end{align*}
% \end{boks}
% \vspace{-5mm}
% \begin{proof}
%   Define the indicators
%   \begin{align*}
%     I_k =
%     \begin{cases}
%       1 \qquad \text{if we get $A$ in the $k$th trial} \\
%       0 \qquad \text{otherwise}
%     \end{cases}
%   \end{align*}
% for $k = 1, 2, \ldots, n$. Then the $I_k$ are i.i.d. and we know from Section 2.5.1 that they have mean $\mu = p$. Since $f_n$ is the sample mean of the $I_k$, the law of large numbers gives $f_n \xrightarrow{P} p$ as $n \rightarrow \infty$.
% \end{proof}

% In the previous section, we saw that $\bar{X} \xrightarrow{P} \mu$ for $n \rightarrow \infty$ or, in other words, that $\bar{X} \equiv \mu$ if $n$ is large.
% It would be good to also have an idea of how accurate this approximation is, that is, have an idea of the extent to which $\bar{X}$ tends to deviate from $\mu$.
% The next result gives the answer.



\begin{boks}{Theorem 4.2 (The Central Limit Theorem)}
  Let $X_1, X_2, \ldots$ be i.i.d. random variables with mean $\mu$ and variance $\sigma^2 < \infty$ and let $S_n = \sum_{k = 1}^{n} X_k$. Then, for each $x \in \mathbb{R}$, we have
  \begin{align*}
    P\left(\frac{S_n - n\mu}{\sigma \sqrt{n}} \leq x \right) \quad \xrightarrow[n \rightarrow \infty]{d} \quad \Phi(x)
  \end{align*}
  as $n \rightarrow \infty$, where $\Phi$ is the cdf of the standard normal distribution.
\end{boks}
In other words
\begin{align*}
  \sqrt{n}\frac{\bar{X}_n - \mu}{\sigma} \quad \xrightarrow[n \rightarrow \infty]{d} \quad N(0,1)
\end{align*}
Thus, the central limit theorem states that the sum of i.i.d. random variables has an approximate normal distribution \textit{regardless of the distribution of the $X_k$}!
The $X_k$ do not even have to be continuous random variables; if we just add enough random variables, the sum becomes approximately normal anyway.

\begin{proof}
  We will only outline the proof and skip the details. The main idea is to work with moment generating functions instead of working directly with the distribution functions. Thus, we will show that the moment generating function of $(S_n - n \mu)/\sigma \sqrt{n}$ converges to the moment generating function of a standard normal distribution.

  Let us first assume that $\mu = 0$ and $\sigma^2 = 1$.
  Let $Y_n = S_n/\sqrt{n}$, and let $M(t)$ be the mgf of the $X_k$. Combining Corollary 3.14 and Proposition 3.40, we see that $Y_n$ has the mgf
  \begin{align}\label{4.1}
    M_{Y_n}(t) = \bigg( M\bigg( \frac{t}{\sqrt{n}} \bigg) \bigg)^n
  \end{align}
  Now do a taylor expansion of $M(s)$ around $s = 0$ to obtain
  \begin{align*}
    M(s) = M(0) + sM'(0) + s^2\frac{M''(0)}{2}
  \end{align*}
  where we hae neglected the error term. By Corollary 3.15, we get
  \begin{align*}
    M(s) = 1 + \frac{s^2}{2}
  \end{align*}
  (remember that $\mu = 0$ and $\sigma^2 = 1$).
  By inserting this into \eqref{4.1}, we obtain
  \begin{align*}
    M_{Y_n}(t) = \bigg( 1 + \frac{t^2}{2n} \bigg)^n \rightarrow e^{t^2/2}
    \quad \text{as} \quad
    n \rightarrow \infty
  \end{align*}
  which we recognize as from Example 3.66 as the mgf of the stadard normal distribution.
  The result for general $\mu$ and $\sigma^2$ follows by considering the random variables $X^\ast_k = (X_k - \mu)/\sigma$, which have mean $0$ and variance $1$, and we leave it to the reader to finish the proof.

  The details we have omitted include and explanation of why convergence of the mgfs is the same as convergence of the distribution functions (which is a fairly deep result and not exactly a detail) and why the error term in the Taylor expansion can be neglected as $n \rightarrow \infty$.
\end{proof}

\begin{boks}{Example 4.5}
  Consider again bufons needle problem. Recall that the probability that the randomly tossed needle intersects a line is $2/\pi$ and how we argued in the previous section that $2/f_n \xrightarrow{P} \pi$ as $ n \rightarrow \infty$. Let $\hat{\pi}$ denote our estimate of $\pi$ after $n$ tosses:
  \begin{align*}
    \hat{\pi} = \frac{2}{f_n}
  \end{align*}
  As mentioned previously, Buffon himself actually that this experiment be used to estimate $\pi$. Let us say that on some occasion he tossed the needle $1000$ times. What is the probability that he got the estimate correct to two decimals?

  We wish to find
  \begin{align*}
    P(|\hat{\pi} - 3.14| \leq 0.005) = P\left( \frac{2}{3.145} \leq f_n \leq \frac{2}{3.135} \right)
  \end{align*}
  where, by the following equation with $p=2/\pi$
  \begin{align*}
    f_n \stackrel{d}{\approx} N\left( p, \frac{p(1 - p)}{n} \right) \qquad
    f_n \stackrel{d}{\approx} N\left( \frac{2}{\pi}, \frac{2(\pi - 2)}{n\pi^2} \right)
  \end{align*}
  With $n = 1000$ we now get
  \begin{align*}
    P(|\hat{\pi} - 3.14| \leq 0.005) &= P \left( \frac{2}{3.145} \leq f_{1000} \leq \frac{2}{3.135} \right)\\
    &=\Phi \left( \frac{2/3.135 - 2/\pi}{\sqrt{2(\pi - 2)/1000\pi^2}} \right) - \Phi\left( \frac{2/3.145 - 2/\pi}{\sqrt{2(\pi - 2)/1000\pi^2}} \right)\\
    &= \Phi(0.09) - \Phi(-0.05) \approx 0.06
  \end{align*}
  i.e. the probability of being correct within two decimals after $1000$ throws is 6 percent. That is no a very good reward for all that needle tossing.
\end{boks}
% \begin{boks}{Definition 4.1}
%   Let $X_1, X_2, \ldots$ be a sequence of discrete random variables such that $X_n$ has pmf $p_{X_n}$. If $X$ is a discrete random variable with pmf $p_X$ and
%   \begin{align*}
%     p_{X_n}(x) \rightarrow p_{X}(x) \quad
%     \text{as} \quad n \rightarrow \infty \quad \text{for all} \quad x
%   \end{align*}
%   then we say that $X_n$ \textit{converges in distribution} to $X$, written $X_n \xrightarrow{d} X$.
% \end{boks}
% \begin{boks}{Proposition 4.2}
%   Let $X_1, X_2, \ldots$ be a sequence of random variables such that $X_n \sim bin(n, p_n)$, where $np_n \rightarrow \lambda > 0$ as $n \rightarrow \infty$, ans let $X \sim \poi(\lambda)$. Then $X_n \xrightarrow{d} X$.
% \end{boks}

% \subsection{Continuous Limits}
% Let us next consider the case when the limiting random variable is continuous. As we already know from the de Moivre-Laplace theorem, the limit can be continuous even if the random variables themselves are not.
% \begin{boks}{Definition 4.2}
%   Let $X_1, X_2, \ldots$ be a sequence of random variables such that $X_n$ has cdf $F_n$. If $X$ is a continuous random variable with cdf $F$ and
%   \begin{align*}
%     F_n(x) \rightarrow F(x)
%     \quad \text{as} \quad
%     n \rightarrow \infty
%     \quad \text{for all} \quad
%     x \in \mathbb{R}
%   \end{align*}
% we say that $X_n$ \textit{converges in distribution} to $X$, written $X_n \xrightarrow{d} X$.
% \end{boks}
% The most important result of this type is the central limit theorem. Another class of important results regarding convergence in distribution deals with the so-called \textit{extreme values}, for example, the minimum and maximum in a sequence of random variables.
