% !TEX root = ../main.tex
\subsection{Discrete Random Variable}
\begin{boks}{Definition 2.1}
A random variable is a real random variable that gets its values from a random experiment.
\begin{align*}
    X: \ S \rightarrow \mathbb{R}.
\end{align*}
\end{boks}
\begin{boks}{Definition 2.2}
If the range of $X$ is countable, then $X$ is called a \textit{discrete random variable}.
\end{boks}

\subsection{Probability Mass Function}
\begin{boks}{Definition 2.3}
Let $X$ be a discrete random variable with range $\{x_1, x_2, \ldots\}$ (finite or countably infinite). The function
\begin{align*}
    p(x_k) = P(X = x_k), \quad k = 1,2,\ldots
\end{align*}
is called the \textit{probability mass function} (pmf) of $X$.
\end{boks}

\begin{minipage}{0.7\textwidth}
  \begin{boks}{Proposition 2.1}
  A function $p$ is a possible pmf of a discrete random variable on the range $\{1,2,\ldots\}$ if and only if
  \vspace{5mm}
  \begin{enumerate}
      \item $p(x_k)\geq 0 \quad \text{for} \ k = 1,2,\ldots$
      \item $\sum_{k=1}^\infty p(x_k) = 1$
  \end{enumerate}
  \end{boks}
\end{minipage}

% \subsection{Cumulative Distribution Function}
% \begin{minipage}{0.7\textwidth}
%   \begin{boks}{Definition 2.4}
%   Let $X$ be any random variable. The function
%   \begin{align*}
%       F(x) = P(X\leq x), \quad \text{for} \ x\in \mathbb{R},
%   \end{align*}
%   is called the (\textit{cumulative}) \textit{distribution function} (cdf) of $X$.
%   \end{boks}
% \end{minipage}
%
% \subsection{Expected Value}
% \begin{boks}{Defintion 2.8}
% Let $X$ be a discrete random variable with range $\{x_1,x_2, \ldots\}$ (finite or countably infinite) and probability mass function $p$. The \textit{expected value} of $X$ is defined as
% \begin{align*}
%     E[X] = \sum_{k=1}^\infty x_k p(x_k).
% \end{align*}
% \end{boks}

\begin{boks}{Proposition 2.12}
  Let $X$ be a random variable with pmf $p_X$ and let $g \ : \ \mathbb{R} \rightarrow \mathbb{R}$ be any function. Then
  \begin{align*}
    E[g(X)] = \sum_{k = 1}^\infty g(x_k)p_X(x_k) \quad \text{if $X$ is discrete with range} \ \{x_1, x_2, \ldots\}
  \end{align*}
\end{boks}

% \begin{boks}{Proposition 2.9}
% Let $X$ be a discrete random variable with range $\{0,1,\ldots\}$. Then
% \begin{align*}
%     E[X] = \sum_{n=0}^\infty P(X > n)
% \end{align*}
% \end{boks}
%
% \begin{proof}
% Note that $k= \sum_{n=1}^k 1$, and use the definition of expected value to obtain
% \begin{align*}
%     E[X] &= \sum_{k=1}^\infty kP(X = k) = \sum_{k=1}^\infty \sum_{n=1}^k P(X = k) \\
%     &= \sum_{n=1}^\infty \sum_{k=n}^\infty P(X = k) = \sum_{n=1}^\infty P(X \geq n)\\
%     &= \sum_{n=0}^\infty P(X > n)
% \end{align*}
% \end{proof}

\begin{boks}{Proposition 2.11 (Linearity for the Expectation)}
Let $X$ be any random variable, and let $a$ and $b$ be real numbers. Then
\begin{align*}
    E[aX + b] = aE[X] + b
\end{align*}
\end{boks}

\begin{proof}
We prove this in the discrete case, for $a > 0$. Let $Y = aX + b$, and note that $Y$ is a discrete random variable and by definition 2.8 and the following
\begin{align*}
  p_Y(y_k) =
  P(Y = y_k) =
  P(aX + b = y_k) =
  P\left( X = \frac{y_k - b}{a} \right) =
  p_X\left( \frac{y_k - b}{a} \right)
\end{align*}
we find that the expectation of $Y$ can be rewritten as follows
\begin{align*}
    E[Y] &= \sum_{k=1}^\infty y_kp_Y(y_k) = \sum_{k=1}^\infty y_kp_X\bigg(\frac{y_k - b}{a}\bigg)\\
    &= \sum_{k=1}^\infty (ax_k + b) p_X(x_k) \\
    &= a\sum_{k=1}^\infty x_kp_X(x_k) + b \sum_{k=1}^\infty p_X(x_k)\\
    &= aE[X] + b
\end{align*}
and we are done.
\end{proof}

\begin{boks}{Definition 2.13}
  If $X$ has probability mass function
  \begin{align*}
    p(k) = \binom{n}{k} p^k (1 - p)^{n - k}, \quad k = 0, 1, \ldots, n
  \end{align*}
  it is said to a \textit{binomial distribution} with parameters $n$ and $p$, and we write $X \sim bin(n,p)$
\end{boks}

\begin{boks}{Proposition 2.17}
  If $X \sim bin(n, p)$, then
  \begin{align*}
    E[X] = np \quad \text{and} \quad \var[X] = np(1 - p)
  \end{align*}
\end{boks}
\begin{proof}
  First note that
  \begin{align*}
    k \binom{n}{k}  = k \frac{n!}{(n - k)! \ k!}
                    = n \frac{(n - 1)!}{(n - 1 - (k - 1))! (k - 1)!}
                    = n \binom{n - 1}{k - 1}
  \end{align*}
  and we get
  \begin{align*}
    k p(k)  = k \binom{n}{k} p^k (1 - p)^{n - k}
            = n \binom{n - 1}{k - 1} p^k (1 - p)^{n - k}
            = np \binom{n - 1}{k - 1} p^{k - 1} (1 - p)^{n - k}
  \end{align*}
  \begin{align*}
    E[X] = \sum_{k = 0}^n kp(k) &= np \sum_{k = 1}^n \binom{n - 1}{k - 1}p^{k - 1}(1 - p)^{n - k}\\
    &= np \sum_{k = 0}^{n - 1} \overbrace{\binom{n - 1}{k} p^k (1 - p)^{n - 1 - k}}^{bin(n - 1, p)} \\
    &= np
  \end{align*}
  since the terms in the sum are the probabilities in a binomial distribution with parameters $n - 1$, and $p$ and hence the sum equals 1.
  The proof of the variance formula is left for problem 65.
\end{proof}
