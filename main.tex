\documentclass{article}

\usepackage[utf8]{inputenc}
\usepackage[english]{babel}
\usepackage{lipsum}
\usepackage{tikz}
\usetikzlibrary{arrows,shapes,positioning}
\usetikzlibrary{calc,decorations.markings}
\usepackage{enumitem}
\setlist{label=(\alph*), topsep=0cm}
\usepackage{amsmath, amssymb, amsthm}
\makeatletter
\renewcommand*\env@matrix[1][*\c@MaxMatrixCols c]{%
  \hskip -\arraycolsep
  \let\@ifnextchar\new@ifnextchar
  \array{#1}}
\makeatother
\usepackage{graphicx}
\usepackage{hyperref}
\usepackage[bb=boondox]{mathalfa}
\usepackage{xcolor}
\usepackage{chngcntr}
\usepackage{mathtools}
\counterwithin{equation}{section}
\renewcommand{\qedsymbol}{$\blacksquare$}

\newcommand\independent{\protect\mathpalette{\protect\independenT}{\perp}}
\def\independenT#1#2{\mathrel{\rlap{$#1#2$}\mkern2mu{#1#2}}}

\newcommand{\upint}[2]{
  \overline{\int_{#1}^{#2}}
}
\newcommand{\loint}[2]{
  \underline{\int_{#1}^{#2}}
}

% Margen
\usepackage[left=2.5cm, right=2.5cm, top=3cm, bottom=3cm]{geometry}

% Eksempel Enviroment
\newcounter{eksempel}[section]
\newenvironment{eksempel}[1][]{\refstepcounter{eksempel}\par\medskip
   \noindent \textbf{Eksempel~\theeksempel. #1} \rmfamily}{\medskip}

% Spacing omkring align enviroment
% \usepackage{etoolbox}
% \newcommand{\zerodisplayskips}{%
%   \setlength{\abovedisplayskip}{-8pt}%       % Spacing over Align
%   \setlength{\belowdisplayskip}{2pt}%        % Spacing under Align
%   \setlength{\abovedisplayshortskip}{0pt}%   % Spacing over Equation
%   \setlength{\belowdisplayshortskip}{0pt}}%  % Spacing under Equation
% \appto{\normalsize}{\zerodisplayskips}%
% \appto{\small}{\zerodisplayskips}%
% \appto{\footnotesize}{\zerodisplayskips}%

%	¤¤ Afsnitsformatering ¤¤ %
\setlength{\parindent}{0mm}           		% Stoerrelse af indryk
\setlength{\parskip}{3mm}          			% Afstand mellem afsnit ved brug af double Enter
\linespread{1,1}							% Linie afstand

\usepackage{mdframed}
\mdtheorem[%
linecolor=black,
  linewidth=.5pt,
  backgroundcolor=gray!10,
  frametitlerule=true,
  frametitlefont=\bfseries,
  frametitlebackgroundcolor=gray!25,
  innertopmargin=\topskip]{theorem}{Theorem}[section]

\mdtheorem[%
linecolor=black,
  linewidth=.5pt,
  backgroundcolor=gray!10,
  frametitlerule=true,
  frametitlefont=\bfseries,
  frametitlebackgroundcolor=gray!25,
  innertopmargin=\topskip]{corollary}[theorem]{Korollar}

\mdtheorem[%
linecolor=black,
  linewidth=.5pt,
  backgroundcolor=gray!10,
  frametitlerule=true,
  frametitlefont=\bfseries,
  frametitlebackgroundcolor=gray!25,
  innertopmargin=\topskip]{lemma}[theorem]{Lemma}

\mdtheorem[%
linecolor=black,
  linewidth=.5pt,
  backgroundcolor=gray!10,
  frametitlerule=true,
  frametitlefont=\bfseries,
  frametitlebackgroundcolor=gray!25,
  innertopmargin=\topskip]{proposition}[theorem]{Proposition}

\mdtheorem[%
linecolor=black,
  linewidth=.5pt,
  backgroundcolor=gray!10,
  frametitlerule=true,
  frametitlefont=\bfseries,
  frametitlebackgroundcolor=gray!25,
  innertopmargin=\topskip]{definition}[theorem]{Definition}

%   Stort spørgsmålstegn kommando
\newcommand{\?}{
\begin{center}
    \huge{\textbf{?}}
\end{center}
}

% Math in section
\newcommand{\mbf}[1]{$\mathbf{#1}$}

%kasse til sætninger mv.
\newenvironment{boks}[1]{\begin{mdframed}\textbf{#1.}}{\end{mdframed}}

\DeclareMathOperator{\poi}{Poi}
\DeclareMathOperator{\unif}{unif}
\DeclareMathOperator{\var}{Var}

\begin{document}

\begin{center}
\thispagestyle{empty}
\parskip=14pt%
\vspace*{3\parskip}%

\begin{figure}[h]
\centering
    \includegraphics[scale=2]{Figures/AAU_logo/AAU_LOGO_RGB.png}
\end{figure}

\vspace{-8pt}
\rule{16cm}{1pt}
\vspace{5pt}

\Huge{\textbf{Exam notes}}\\
\vspace{8pt}
\Large{\textbf{4th semester:} \textit{Probability theory}}\\
\vspace{50pt}

\Large{\textbf{Kasper Rosenkrands}}
\end{center}

\newpage

\pagenumbering{roman}
\renewcommand{\baselinestretch}{0.95}\normalsize
\tableofcontents
\renewcommand{\baselinestretch}{1.0}\normalsize
\newpage

\pagenumbering{arabic}

\section{Basics of probability (incl. combinatorics, law of total probability and Bayes' formula).}
% !TEX root = ../main.tex
\subsection{Axioms of Probability}
\begin{boks}{Definition 1.3 (Axioms of Probability)}
  A \textit{probability measure} is a function $P$, which assigns to each event $A$ a number of $P(A)$ satisfying
  \begin{enumerate}[label = \textbf{(\alph*)}]
    \item $0 \leq P(A) \leq 1$
    \item P(S) = 1
    \item If $A_1, A_2, \ldots$ is a sequence of \textit{pairwise disjoint} events, that is, if $i \neq j$, then $A_i \cap A_j = \emptyset$, then
    \begin{align*}
      P\left( \bigcup_{k = 1}^\infty A_k \right) = \sum_{k = 1}^\infty P(A_k)
    \end{align*}
  \end{enumerate}
\end{boks}
\subsection{Conditional Probability}
\begin{boks}{Definition 1.4}
  Let $B$ be an event such that $P(B) > 0$. For any event $A$, denote and define the \textit{conditional probability} of $A$ \textit{given} $B$ as
  \begin{align*}
    P(A|B) &= \frac{P(A \cap B)}{P(B)}, \quad \text{or}\\
    P(A|B)P(B) &= P(A \cap B)
  \end{align*}
\end{boks}

\begin{table}[h]
\centering
\begin{tabular}{l|ll}
 & With Replacement & Without Replacement \\ \hline
With regard to order & \multicolumn{1}{l|}{$n^k$} & \multicolumn{1}{l|}{$n(n - 1)\cdots(n - k - 1) = \dfrac{n!}{(n - k)!} = (n)_k$} \\ \cline{2-3}
Without regard to order & \multicolumn{1}{l|}{$\binom{n - 1 - k}{k}$} & \multicolumn{1}{l|}{$\binom{n}{k} = \dfrac{n!}{(n - k)!}$} \\ \cline{2-3}
\end{tabular}
\end{table}

\subsection{Law of Total Probability}
\begin{boks}{Theorem 1.1 (Law of Total Probability)}
    Let $B_1, B_2, \ldots$ be a sequence of events such that
    \begin{enumerate}
        \item $P(B_k) > 0$ for $k = 1, 2, \ldots$
        \item $B_i$ og $B_j$ are disjoint whenever $i \neq j$
        \item $S = \bigcup_{k=1}^\infty B_k$
    \end{enumerate}
    Then, for any event $A$, we have
    \begin{align*}
        P(A) = \sum_{k=1}^\infty P(A|B_k)P(B_k).
    \end{align*}
\end{boks}

\begin{proof}
First note that
\begin{align*}
    A = A \cap S = \bigcup_{k = 1}^\infty (A \cap B_k),
\end{align*}
by the distributive law for infinite unions. Since $A \cap B_1, A \cap B_2, \ldots$ are pairwise disjoint,  we get
\begin{align*}
    P(A) = P \left(\bigcup_{k = 1}^\infty (A \cap B_k)\right) = \sum_{k = 1}^\infty P(A \cap B_k) = \sum_{k = 1}^\infty P(A|B_k)P(B_k).
\end{align*}
Which proves the theorem. Note that the result also holds for finite sequences.
\end{proof}

\begin{boks}{Corollary 1.6}
If $0 < P(B) < 1$, for $B\subseteq S$, then
\begin{align*}
    P(A) = P(A|B)P(B) + P(A|B^c)P(B^c)
\end{align*}
\end{boks}

\subsection{Bayes Formula}
\begin{boks}{Proposition 1.11 (Bayes' Formula)}
Under the same assumptions as in the law of total probability and if $P(A) > 0$, then for any event $B_j$, we have
\begin{align*}
    P(B_j|A) &= \frac{P(A|B_j)P(B_j)}{\sum_{k=1}^\infty P(A|B_k)P(B_k)} \\
    &= \frac{P(A|B_j)P(B_j)}{P(A)} \qquad \text{according to the LTP.}
\end{align*}
\end{boks}
\begin{proof}
  Note that, by the law of total probability, the denominator is nothing but $P(A)$, and hence we must show that
  \begin{align*}
    P(B_j|A) = \frac{P(A|B_j)P(B_j)}{P(A)}
  \end{align*}
  which is to say that
  \begin{align*}
    P(B_j|A)P(A) = P(A|B_j)P(B_j)
  \end{align*}
  which is true since both sides equal $P(A \cap B_j)$, by the definition of conditional probability.
\end{proof}

\begin{boks}{Corollary 1.7}
  If $0 < P(B) < 1$ and $P(A) > 0$, then
  \begin{align*}
    P(B|A) = \frac{P(A|B)P(B)}{P(A|B)P(B) + P(A|B^c)P(B^c)}
  \end{align*}
\end{boks}

\newpage

\section{Discrete stochastic variables and distributions (incl. means and variances).}
% !TEX root = ../main.tex
\subsection{Discrete Random Variable}
\begin{boks}{Definition 2.1}
A random variable is a real random variable that gets its values from a random experiment.
\begin{align*}
    X: \ S \rightarrow \mathbb{R}.
\end{align*}
\end{boks}
\begin{boks}{Definition 2.2}
If the range of $X$ is countable, then $X$ is called a \textit{discrete random variable}.
\end{boks}

\subsection{Probability Mass Function}
\begin{boks}{Definition 2.3}
Let $X$ be a discrete random variable with range $\{x_1, x_2, \ldots\}$ (finite or countably infinite). The function
\begin{align*}
    p(x_k) = P(X = x_k), \quad k = 1,2,\ldots
\end{align*}
is called the \textit{probability mass function} (pmf) of $X$.
\end{boks}

\begin{minipage}{0.7\textwidth}
  \begin{boks}{Proposition 2.1}
  A function $p$ is a possible pmf of a discrete random variable on the range $\{1,2,\ldots\}$ if and only if
  \vspace{5mm}
  \begin{enumerate}
      \item $p(x_k)\geq 0 \quad \text{for} \ k = 1,2,\ldots$
      \item $\sum_{k=1}^\infty p(x_k) = 1$
  \end{enumerate}
  \end{boks}
\end{minipage}

\subsection{Cumulative Distribution Function}
\begin{minipage}{0.7\textwidth}
  \begin{boks}{Definition 2.4}
  Let $X$ be any random variable. The function
  \begin{align*}
      F(x) = P(X\leq x), \quad \text{for} \ x\in \mathbb{R},
  \end{align*}
  is called the (\textit{cumulative}) \textit{distribution function} (cdf) of $X$.
  \end{boks}
\end{minipage}

\subsection{Expected Value}
\begin{boks}{Defintion 2.8}
Let $X$ be a discrete random variable with range $\{x_1,x_2, \ldots\}$ (finite or countably infinite) and probability mass function $p$. The \textit{expected value} of $X$ is defined as
\begin{align*}
    E[X] = \sum_{k=1}^\infty x_k p(x_k).
\end{align*}
\end{boks}

\begin{boks}{Proposition 2.12}
  Let $X$ be a random variable with pmf $p_X$ and let $g \ : \ \mathbb{R} \rightarrow \mathbb{R}$ be any function. Then
  \begin{align*}
    E[g(X)] = \sum_{k = 1}^\infty g(x_k)p_X(x_k) \quad \text{if $X$ is discrete with range} \ \{x_1, x_2, \ldots\}
  \end{align*}
\end{boks}

% \begin{boks}{Proposition 2.9}
% Let $X$ be a discrete random variable with range $\{0,1,\ldots\}$. Then
% \begin{align*}
%     E[X] = \sum_{n=0}^\infty P(X > n)
% \end{align*}
% \end{boks}
%
% \begin{proof}
% Note that $k= \sum_{n=1}^k 1$, and use the definition of expected value to obtain
% \begin{align*}
%     E[X] &= \sum_{k=1}^\infty kP(X = k) = \sum_{k=1}^\infty \sum_{n=1}^k P(X = k) \\
%     &= \sum_{n=1}^\infty \sum_{k=n}^\infty P(X = k) = \sum_{n=1}^\infty P(X \geq n)\\
%     &= \sum_{n=0}^\infty P(X > n)
% \end{align*}
% \end{proof}

\begin{boks}{Proposition 2.11 (Linearity for the Expectation)}
Let $X$ be any random variable, and let $a$ and $b$ be real numbers. Then
\begin{align*}
    E[aX + b] = aE[X] + b
\end{align*}
\end{boks}

\begin{proof}
We prove this in the discrete case, for $a > 0$. Let $Y = aX + b$, and note that $Y$ is a discrete random variable and by definition 2.8 expected value
\begin{align*}
    E[Y] &= \sum_{k=1}^\infty y_kp_Y(y_k) = \sum_{k=1}^\infty y_kp_X\bigg(\frac{y_k - b}{a}\bigg)\\
    &= \sum_{k=1}^\infty (ax_k + b) p_X(x_k) \\
    &= a\sum_{k=1}^\infty x_kp_X(x_k) + b \sum_{k=1}^\infty p_X(x_k)\\
    &= aE[X] + b
\end{align*}
and we are done.
\end{proof}

\newpage

\section{Continuous stochastic variables and distributions (incl. means and variances).}
% !TEX root = ../main.tex
\subsection{Continuous Stochastic Variable}
\begin{boks}{Definition 2.5}
If the cdf $F$ is a continuous and differentiable function, then $X$ is said to be a \textit{continuous random variable}.
\end{boks}

\subsection{Cumulative Distribution Function}
\begin{boks}{Proposition 2.3}
If $F$ is the cdf of any random variable, $F$ has the following properties:
\begin{enumerate}
    \item It is nondecreasing
    \item It is right-continuous
    \item It has the limits $F(-\infty) = 0$ and $F(\infty) = 1$ (where the limits may or may not be attained at finite $x$).
\end{enumerate}
\end{boks}

% \begin{boks}{Proposition 2.4}
% Let $X$ be any random variable with cdf $F$. Then
% \begin{enumerate}
%     \item $P(a < X \leq b) = F(b) - F(a), \quad a \leq b$
%     \item $P(X > x) = 1 - F(x), \quad x \in \mathbb{R}$
%     \item $F$ is continuous from the right with limit form the left and increasing
% \end{enumerate}
% \end{boks}

\subsection{Probability Density Function}
\begin{boks}{Definition 2.6}
The function $f(x) = F'(x)$ is called the \textit{probability density function} (pdf) of $X$.
\end{boks}
\begin{minipage}{0.7\textwidth}
  \begin{boks}{Proposition 2.5}
  Let X be a continuous random variable with pdf $f$ and cdf $F$. Then
  \vspace{5mm}
  \begin{enumerate}
      \item $F(x) = \int_{-\infty}^x f(t)dt, \quad x \in \mathbb{R}$
      \item $f(x) = F'(x), \quad x \in \mathbb{R}$
      \item For $B \subseteq \mathbb{R}, \quad P(X\in B) = \int_B f(x)dx$
  \end{enumerate}
  \end{boks}
\end{minipage}
\begin{boks}{Proposition 2.6}
A function $f$ is a possible pdf of some continuos random variable if and only if
\begin{enumerate}
    \item $f(x) \geq 0, \quad x \in \mathbb{R}$
    \item $\int_{-\infty}^\infty f(x)dx = 1$
\end{enumerate}
\end{boks}

\subsection{Proposition 2.8}
\begin{boks}{Proposition 2.8}
Let $X$ be a continuous random variable with pdf $X$, let $g$ be a strictly increasing or strictly decreasing, differentiable function, and let $Y = g(X)$. Then $Y$ has pdf
\begin{align*}
    f_Y(y) = \bigg| \frac{d}{dy}g^{-1}(y)\bigg| f_X(g^{-1}(y))
\end{align*}
for $y$ in range of $Y$.
\end{boks}
\begin{proof}
  \begin{align*}
      F_Y(y)&=P(Y\leq y)=P(g(X)\leq y)=P(X\leq g^{-1}(y))=F_X(g^{-1}(y))
  \end{align*}
  \begin{align*}
      f_Y(y)&=F'_Y(y)=F'_X(g^{-1}(y))= \frac{d}{dy}g^{-1}(y)\cdot F'_X(g^{-1}(y))=\frac{d}{dy}g^{-1}(y)\cdot f_X(g^{-1}(y))
  \end{align*}
\end{proof}

\subsection{Expected Value}
\begin{boks}{Defintion 2.9}
Let $X$ be a continuous random variable with pdf $f$. The \textit{expected value} of $X$ is defined as
\begin{align*}
    E[X] = \int_{-\infty}^\infty xf(x)dx = \int_\mathbb{R} xf(x)dx,
\end{align*}
notice that the last equality is not always satisfied, but for the purpose of this course it is.
\end{boks}

% \begin{boks}{Proposition 2.10}
% Let $X$ be a continuous random variable with range $[0, \infty)$. Then
% \begin{align*}
%     E[x] = \int_0^\infty P(X > x)dx
% \end{align*}
% \end{boks}

% \begin{boks}{Proposition 2.11 (Linearity for the Expectation)}
% Let $X$ be any random variable, and let $a$ and $b$ be real numbers. Then
% \begin{align*}
%     E[aX + b] = aE[X] + b
% \end{align*}
% \end{boks}
%
% \begin{proof}
% We prove this in the continuous case, for $a > 0$. Let $Y = aX + b$, and note that $Y$ is a continuous random variable, which by proposition 2.8 has pdf
% \begin{align*}
%     f_Y(y) = \frac{1}{a} f_X\bigg(\frac{y-b}{a}\bigg)
% \end{align*}
% and by definition, the expected value of $Y$ is
% \begin{align*}
%     E[Y] = \int_{-\infty}^\infty yf_Y(y)dy = \frac{1}{a}\int_{-\infty}^\infty yf_X\bigg(\frac{y-b}{a}\bigg)dy
% \end{align*}
% where the variable $y=ax+b$ gives $dy = a \ dx$ and hence
% \begin{align*}
%     E[Y] &= \int_{-\infty}^\infty (ax + b)f_X(x)dx \\
%     &= a \int_{-\infty}^\infty xf_X(x)dx + b \int_{-\infty}^\infty f_X(x)dx \\
%     &= aE[X] + b
% \end{align*}
% and we are done.
% \end{proof}

\subsection{Variance}
\begin{boks}{Definition 2.10}
Let $X$ be a random variable with expected value $\mu$. The \textit{variance} of $X$ is defined as
\begin{align*}
    Var[X] = E\Big[ (X - \mu)^2 \Big]
\end{align*}
\end{boks}

% \begin{boks}{Definition 2.11}
% Let $X$ be a random variable with variance $\sigma^2 = Var[X]$. The \textit{standard deviation} of $X$ is then defined as $\sigma = \sqrt{Var[X]}$.
% \end{boks}

\begin{boks}{Corollary 2.2}
\begin{align*}
    Var[X] = E[X^2] - (E[X])^2
\end{align*}
\end{boks}

\begin{proof}
By proposition 2.12, we have
\begin{align*}
    Var[X] &= E\Big[(X-\mu)^2\Big] = \int_{-\infty}^\infty (x - \mu)^2 dx\\
    &= \int_{-\infty}^\infty (x^2 - 2x\mu + \mu^2) f(x) dx \\
    &= \int_{-\infty}^\infty x^2 f(x) dx - 2\mu \int_{-\infty}^\infty x f(x) dx + \mu^2\int_{-\infty}^\infty f(x)dx\\
    &=E[X^2] - 2\mu E[X] + \mu^2\\
    &= E[X^2] - 2 E[X]^2 + E[X]^2\\
    &= E[X^2] - E[X]^2.
\end{align*}
\end{proof}

% \begin{boks}{Proposition 2.14 (Chebyshev's Inequality)}
% Let $X$ be any random variable with mean $\mu$ and variance $\sigma^2$. For any consatn $c > 0$, we have
% \begin{align*}
%     P(|X - \mu| \geq c\sigma) \leq \frac{1}{c^2}.
% \end{align*}
% \end{boks}
%
% \begin{proof}
% Let us prove the continuous case. Fix $c$ and let $B$ be the set $\{x \in \mathbb{R} : |x - \mu| \geq c\sigma\}$. We get
% \begin{align*}
%     \sigma^2 &= E[(X - \mu)^2] = \int_{-\infty}^\infty (x-\mu)^2f(x)dx=\int_\mathbb{R}(x-\mu)^2f(x)dx\\
%     &\geq \int_B(x-\mu)^2f(x)dx \geq c^2\sigma^2\int_Bf(x)dx = c^2\sigma^2P(X\in B).
% \end{align*}
% Which gives the desired inequality.
% \end{proof}

\newpage

\section{Two random variables: select from topics such as joint distribution, conditional distribution, independence and convolution.}
% !TEX root = ../main.tex
\begin{boks}{Definition 3.1}
Let $X$ and $Y$ be random variables. The pair $(X,Y)$ is then called a (two-dimensional) \textit{random vector.}
\end{boks}

\begin{boks}{Definition 3.2}
The \textit{joint distribution function} (joint cdf) of $(X,Y)$ is defined as
\begin{align*}
    F(x,y) = P(X \leq x, Y \leq y)
\end{align*}
for $x,y \in \mathbb{R}$.
\end{boks}

\begin{boks}{Proposition 3.1 (Marginal cdf)}
If $(X, Y)$ has joint cdf $F$, then $X$ and $Y$ have cdfs
\begin{align*}
    F_X(x) = F(x, \infty) \quad \text{and} \quad F_Y(y) = F(\infty, y)
\end{align*}
for $x,y \in \mathbb{R}$. Notice that there is a slight abuse of notation here, $F(x,\infty)$ refers to $\lim_{y \rightarrow \infty} F(x, y)$.
\end{boks}

\begin{boks}{Definition 3.5}
If there exists a function $f$ such that
\begin{align*}
    P((X, Y)\in B) = \int\int_B f(x, y) dx dy
\end{align*}
for all subsets $B\subseteq \mathbb{R}^2$, then $X$ adn $Y$ are said to be \textit{jointly continuous}. The function $f$ is calle the \textit{joint pdf}.
\end{boks}

\begin{boks}{Propostion 3.3}
If $X$ and $Y$ are jointly continuous with joint cdf $F$ adn joint pdf $f$, then
\begin{align*}
    f(x, y) = \frac{\partial^2}{\partial x \partial y} F(x,y), \quad x,y \in \mathbb{R}.
\end{align*}
\end{boks}

\begin{boks}{Proposition 3.4}
A function $f$ is a possible joint pdf for the random variables $X$ and $Y$ if and only if
\begin{enumerate}
    \item $f(x, y) \geq 0$ for all $x,y \in \mathbb{R}$
    \item $\int_{-\infty}^\infty \int_{-\infty}^\infty f(x,y)dx\  dy = 1$
\end{enumerate}
\end{boks}

\begin{boks}{Proposition 3.5}
Suppose that $X$ and $Y$ are jointly continuous with joint pdf $f$. Then $X$ adn $Y$ are continuous random variables with marginal pdfs
\begin{align*}
    f_X(x) &= \int_{-\infty}^\infty f(x, y) dy, \quad x\in\mathbb{R} \\
    f_Y(y) &= \int_{-\infty}^\infty f(x,y) dx, \quad y\in\mathbb{R}.
\end{align*}
\end{boks}

\begin{boks}{Definition 3.7}
Let $(X, Y)$ ne jointly continuous with joint pdf $f$. The \textit{conditional pdf of $Y$ given $X=x$} is defined as
\begin{align*}
    f_Y(y|x) = \frac{f(x,y)}{f_X(x)}, \quad y\in\mathbb{R}.
\end{align*}
\end{boks}

The following proposition is a continuous version of the law of total probability.

\begin{boks}{Proposition 3.6}
Let $X$ and $Y$ be jointly continuous. Then
\begin{enumerate}
    \item
    \begin{align*}
        f_Y(y) = \int_{-\infty}^\infty f_Y(y|x)f_X(x)dx, \quad y\in\mathbb{R}
    \end{align*}
    \item
    \begin{align*}
        P(Y \in B) = \int_{-\infty}^\infty P(Y \in B|X = x)f_X(x)dx, \quad B\subseteq\mathbb{R}.
    \end{align*}
\end{enumerate}
\end{boks}

\begin{proof}
For (a), just combine Proposition 3.5 with the definition of conditional pdf for (b), part (a) gives
\begin{align*}
    P(Y \in B) &= \int_B f_Y(y)dy = \int_B \int_{-\infty}^\infty f_Y(y|x)f_X(x)dx \ dy \\
    &= \int_{-\infty}^\infty \int_B f_Y(y|x)f_X(x)dy \ dx \\
    &= \int_{-\infty}^\infty P(Y \in B|X = x) f_X(x)dx
\end{align*}
as desired.
\end{proof}

\begin{boks}{Proposition 3.10}
Suppose that $X$ and $Y$ are jointly continuous with joint pdf $f$. Then $X$ and $Y$ are independent if and only if
\begin{align*}
    f(x, y) = f_X(x)f_Y(y)
\end{align*}
for all $x,y \in \mathbb{R}$.
\end{boks}

\begin{boks}{Corrollary 3.1}
\textit{The random variables $X$ and $Y$ are independent if and only if ``the joint is the product of the marginals.''}
\end{boks}

\subsection{Convolution}
\begin{boks}{Proposition 3.34}
Let $X$ and $Y$ be independent continuous random variables with pdfs $f_X$ and $f_Y$, respectively. The pdf of the sum $X + Y$ is then
\begin{align*}
    f_{X+Y} = \int_{-\infty}^\infty f_Y(x - u)f_X(u)du, \quad x\in \mathbb{R}.
\end{align*}
\end{boks}

\newpage

\section{Two random variables: select from topics such as covariance and correlation, conditional expectation, conditional variance and the bivariate normal distribution.}
% !TEX root = ../main.tex
\begin{boks}{Definition 3.10}
  Let $y$ be random variable and $B$ an event with $P(B)>0$. The \textit{conditional expectation} of $Y$ given $B$ is defined as
  \begin{align*}
    E[Y|B] =  \begin{cases}
                \sum_{k = 1}^\infty y_k P(Y = y_k|B) & \text{if $Y$ is discrete with range $\{y_1, y_2, \ldots\}$}\\
                \int_{-\infty}^\infty yf_Y(y|B)dy & \text{if $Y$ is continuous}
              \end{cases}
  \end{align*}
\end{boks}

% \begin{boks}{Definition 3.11}
%   Suppose that $X$ and $Y$ are discrete. We define
%   \begin{align*}
%     E[Y|X=x_j] = \sum_{k=1}^\infty y_k p_Y(y_k|x_j).
%   \end{align*}
% \end{boks}

\begin{boks}{Definition 3.12}
  Suppose that $X$ and $Y$ are jointly continuous. We define
  \begin{align*}
    E[Y|X = x] = \int_{-\infty}^\infty y f_Y(y|x)dx.
  \end{align*}
\end{boks}

Following the usual intuitive interpretation, this is the expected value for $Y$ if we know that $X = x$. The law of total expectation now takes the following form.

\begin{boks}{Propostion 3.17}
  Suppose that $X$ and $Y$ are jointly continuous. Then
  \begin{align*}
    E[Y] = \int_{-\infty}^\infty E[Y|X = x] f_X(x)dx.
  \end{align*}
\end{boks}
\begin{proof}
  By definiton of expected value and Proposition 3.6 (a)
  \begin{align*}
    E[Y] =  \int_{-\infty}^\infty y f_Y(y)dy = \int_{-\infty}^\infty \int_{-\infty}^\infty yf_Y(y|x)f_X(x)dx \ dy
  \end{align*}
  where we cahnge the order of integration to obtain
  \begin{align*}
    E[Y] = \int_{-\infty}^\infty\int_{-\infty}^\infty y f_Y(y|x)dy f_X(x)dx
  \end{align*}
  where the inner integral equals $E[Y|X = x]$ by definition, and we are done.
\end{proof}

% \subsection{Conditional Expectation as a Random Variable}
%
% \begin{boks}{Definition 3.13}
%   The \textit{conditional expectation} of $Y$ given $X$, $E[Y|X]$, is a random variable that equals $E[Y|X = x]$ whenever $X = x$.
% \end{boks}
%
% \begin{boks}{Corollary 3.5}
%   \begin{align*}
%     E[Y] = E\Big[E[Y|X]\Big]
%   \end{align*}
% \end{boks}
%
% \begin{boks}{Definition 3.14}
%   Let $g(X)$ be a predictor of $Y$. The \textit{mean square error} is defined as
%   \begin{align*}
%     E\Big[(Y - g(X))^2\Big].
%   \end{align*}
% \end{boks}
%
% \begin{boks}{Proposition 3.18}
%   Among all predictors $g(X)$ of $Y$, the mean square error is minimized by $E[Y|X]$.
% \end{boks}
%
% We omit the proof and instead refer to an intuitive argument. Suppose that we want to predcit $Y$ as much as possible by a constant value $c$. Then, we want to minimize $E[(Y - c)^2]$, and with $\mu = E[Y]$ we get
% \begin{align*}
%   E\Big[(Y - c)^2\Big] &= E\Big[(Y - \mu + \mu - c)^2\Big]\\
%   &= E\Big[(Y - \mu)^2] + 2(\mu - c)E[(Y - \mu)] + (\mu - c)^2\\
%   &= Var[Y] + (\mu - c)^2
% \end{align*}
% since $E[Y - \mu] = 0$. But the last expression is minimized when $\mu = c$ and hence $\mu$ is the best predictor of $Y$ among all constants. This is not too surprising; if we do not know anything about $Y$, the best guess should be the expected value $E[Y]$. Now, if we observe another random variable $X$, the same ought to be true: $Y$ is best predicted by its expected value given the random variable $X$, that is, $E[Y|X]$.
%
\subsection{Conditional Variance}
\begin{boks}{Definition 3.15}
  The \textit{conditional variance} of $Y$ given $X$ is defined as
  \begin{align*}
    Var[Y|X] = E\Big[(Y - E[Y|X])^2|X\Big].
  \end{align*}
\end{boks}

Note that the conditonal varince is also a random variable and we think of it as the variance of $Y$ given the value $X$. In particular,  if we have the observed $X = x$, then we can denote and defined
\begin{align*}
  Var[Y|X = x] = E\Big[(Y - E[Y|X = x])^2|X = x\Big].
\end{align*}
also note that if $X$ and $Y$ are independent, $E[Y|X]=E[Y]$, and the definition boils down to the regular variance. There is an analog of Corollary 2.2, which we leave to the reader to prove.

\begin{boks}{Corollary 3.7}
  \begin{align*}
    Var[Y|X] = E\Big[[Y^2|X]\Big] - (E[Y|X])^2
  \end{align*}
\end{boks}

There is also a ``law of total variance'', which looks slightly more complicated than that of total expectation.

\begin{boks}{Proposition 3.19}
  \begin{align*}
    Var[Y] = Var\Big[ E[Y|X] \Big] + E\Big[Var[Y|X]\Big]
  \end{align*}
\end{boks}
\begin{proof}
  Take expected values in Corollary 3.7 to obtain
  \begin{align}\label{eq:3.1}
    E\Big[ \var[Y|X] \Big] = E[Y^2] - E \Big[ (E[Y|X])^2 \Big]
  \end{align}
  and since $E\Big[E[Y|X]\Big] = E[Y]$, we have
  \begin{align}\label{eq:3.2}
    \var\Big[ E[Y|X] \Big] = E\Big[ (E[Y|X])^2 \Big] - (E[Y])^2
  \end{align}
  and the result follows form adding \eqref{eq:3.1} and \eqref{eq:3.2}.
\end{proof}

\subsection{Covarince and Correlation}

\begin{boks}{Definition 3.16}
  The \textit{covariance} of $X$ and $Y$ is defined as
  \begin{align*}
    Cov[X, Y] = E\Big[(X - E[X])(Y - E[Y])\Big].
  \end{align*}
\end{boks}

\begin{boks}{Proposition 3.20}
  \begin{align*}
    Cov[X, Y] = E[XY] - E[X]E[Y]
  \end{align*}
\end{boks}
%
% \begin{boks}{Corollary 3.8}
%   \textit{If $X$ and $Y$ are independent, then $Cov[X, Y] = 0$.}
% \end{boks}
%
\begin{boks}{Proposition 3.21}
  \begin{align*}
    Var[X + Y = Var[x] + Var[Y] + 2Cov[X, Y].
  \end{align*}
\end{boks}

\begin{proof}
  By the definition of variance and covariance and repeated use of properties of expected values, we get
  \begin{align*}
    Var[X + Y] &= E\Big[(X + Y - E[X + Y])^2]\\
    &= E\Big[(X - E[X] + Y - E[Y])^2]\\
    &= E\Big[(X - E[X])^2 + (Y - E[Y])^2 + 2(X - E[X])(Y - E[Y])]\\
    &= Var[X] + Var[Y] + Cov[X, Y]
  \end{align*}
  and we are done.
\end{proof}

\textbf{Måske kan du vælge bivariate normal i stedet hvis du har tid og føler dig klog!}

% \begin{boks}{Proposition 3.22}
%   Let $X, Y$, and $Z$ be random variables and let $a$ and $b$ be real numbers. Then
%   \begin{enumerate}
%     \item $Cov[X, X] = Var[X]$
%     \item $Cov[aX, bX] = abCov[X, Y]$
%     \item $Cov[X + Y, Z] = Cov[X, Z] + Cov[Y, Z]$
%   \end{enumerate}
%   together these properties indicate that covariance is \textit{bilinear}.
% \end{boks}
%
% \subsection{The Correlation Coefficient}
%
% \begin{boks}{Definition 3.17}
%   The \textit{correlation coefficient} of $X$ and $Y$ is defined as
%   \begin{align*}
%     \rho(X, Y) = \frac{Cov[X, Y]}{\sqrt{Var[X]Var[Y]}}.
%   \end{align*}
% \end{boks}
%
% The correlation coefficient is dimensionless. To demonstrate this, take $a, b > 0$ and note that
% \begin{align*}
%   \rho(aX, bY) &= \frac{Cov[aX, bY]}{\sqrt{Var[aX]Var[bY]}}\\
%   &= \frac{abCov[X, Y]}{\sqrt{a^2Var[X]b^2Var[Y]}} = \rho(X, Y).
% \end{align*}
% We also call $\rho(X, Y)$ simply the correlation between $X$ and $Y$. Here are som good properties of the correlation coefficient.
%
% \begin{boks}{Proposition 3.25}
%   The correlation coefficient of any pair of random variables $X$ and $Y$ satisfies
%   \begin{enumerate}
%     \item $-1\leq\rho(X, Y)\leq1$
%     \item If $X$ and $Y$ are independent, then $\rho(X, Y) = 0$
%     \item $\rho(X, Y) = 1$ if and only if $Y = aX + b$,  where $a > 0$
%     \item $\rho(X, Y) = -1$ if and only if $Y = aX + b$,  where $a < 0$
%   \end{enumerate}
% \end{boks}
%
% \begin{proof}
%   Let $Var[X] = \sigma_1^2$ and $Var[Y] = \sigma_2^2$. For (a), first apply Proposition 3.21 to the random variables $X/\sigma_1$ and $Y/\sigma_2$ and use the properties of the variance and covariance to obtain
%   \begin{align*}
%     0 \leq Var\bigg[ \frac{X}{\sigma_1} + \frac{Y}{\sigma_2} \bigg] = \frac{Var[X]}{\sigma_1} + \frac{Var[Y]}{\sigma_2} + \frac{2Cov[X, Y]}{\sigma_1\sigma_2} = 2 + 2\rho
%   \end{align*}
%   which gives $\rho\geq1$. To show that $\rho\leq1$, instead use $X/\sigma_1$ and $-Y/\sigma_2$. Part (b) follows from Corollary 3.8, and parts (c) and (d) follows from Proposition 2.16, applied to tha random variables $X/\sigma_1 - Y/\sigma_2$ and $X/\sigma_1 + Y/\sigma_2$, respectively. Note that this also gives $a$ and $b$ expressed in terms of the means, variances, and correlation coefficeinet (see problem 90).
% \end{proof}
%
% \subsection{The Bivariate Normal Distribution}
% \begin{boks}{Definition 3.18}
%   If $(X, Y)$ has joint pdf
%   \begin{align*}
%     &f(x,y) = \frac{1}{2\pi\sigma_1\sigma_2\sqrt{1 - \rho^2}}\\
%     &\times \exp \left\{ -\frac{1}{2(1-\rho^2)} \bigg( \frac{(x-\mu_1)^2}{\sigma_1^2} + \frac{(y-\mu_2)^2}{\sigma_2^2} -
%     \frac{2\rho (x-\mu_1)(y-\mu_2)}{\sigma_1\sigma_2} \bigg) \right\}
%   \end{align*}
%   for $x,y \in \mathbb{R}$, then $(X,Y)$ is said to have a \textit{bivariate normal distribution}.
% \end{boks}
% \begin{boks}{Proposition 3.27}
%   Let $(X,Y)$ have a bivariate normal distribution with parameters $\mu_1, \mu_2, \sigma_1, \sigma_2, \rho$. Then
%   \begin{enumerate}
%     \item $X \sim N(\mu_1, \sigma_1^2)$ and $Y \sim N(\mu_2, \sigma_2^2)$
%     \item $\rho$ is the correlation coefficient of $X$ and $Y$
%   \end{enumerate}
% \end{boks}
% \begin{proof}
%   \textbf{mangler}
% \end{proof}
% \begin{boks}{Proposition 3.28}
%   Let $(X, Y)$ be bivariate normal. Then, for fixed $x \in \mathbb{R}$
%   \begin{align*}
%     Y|X = x \sim N \left( \mu_2 + \rho\frac{\sigma_2}{\sigma_1}(x - \mu_1)
%     , \sigma_2^2(1 - \rho^2) \right)
%   \end{align*}
% \end{boks}
% \begin{proof}
%   \textbf{mangler}
% \end{proof}
% \begin{boks}{Proposition 3.29}
%   Let $(X,Y)$ be bivariate normal. Then $X$ and $Y$ are independent if and only if they are uncorrelated.
% \end{boks}
% \begin{proof}
%   \textbf{mangler}
% \end{proof}
%
% Der mangler stadig en masse her... Men der må være en grænse!

\newpage

\section{Generating functions (possibly with a focus on how probability generating functions relate to thinning of a Poisson process).}
% !TEX root = ../main.tex
\textit{Generating functions}, or \textit{transforms}, are very useful in probability theory as in other fields of mathematics. Several different generating functions are used, depending on the type of random variable. We will discuss two, one that is useful for discrete random variables and the other for continuous random variables.

\subsection{The Probability Generating Function}
When we study nonnegative, integer-valued random variables, the following function proves to be a useful tool.

\begin{boks}{Definition 3.23}
  Let $X$ be nonnegative and integer valued. The function
  \begin{align*}
    G_X(s) = E[s^X] = \sum_{k=0}^\infty s^kP(X=k) \quad 0 \leq s \leq 1
  \end{align*}
  is called the \textit{probability generating function} (pgf) of $X$.
\end{boks}

\begin{boks}{Corollary 3.12}
  Let $X$ be a nonnegative and integer valued with pgf $G_X$. then
  \begin{align*}
    G_X(0) = p_X(0) \quad \text{and} \quad G_X(1) = 1
  \end{align*}
\end{boks}

\begin{boks}{Proposition 3.36}
  Let $X$ be nonnegative and integer valued with pgf $G_X$. then
  \begin{align*}
    p_X(k) = \frac{G_X^{(k)}(0)}{k!}, \quad k = 0,1,\ldots
  \end{align*}
  where $G_X^{(k)}(0)$ denotes the $k$th derivative of $G_X$.
\end{boks}

\begin{boks}{Proposition 3.37}
  If $X$ has pgf $G_X$, Then
  \begin{align*}
    E[X] = G'_X(1) \quad \text{and} \quad Var[X] = G''_X(1) + G'_X(1) - G'_X(1)^2
  \end{align*}
\end{boks}

\begin{boks}{Proposition 3.38}
  Let $X_1, X_2, \ldots, X_n$ be independent random variables with pgfs $G_1, G_2, \ldots, G_n$ respectively and let $S_n = X_1 + X_2 + \cdots + X_n$. Then $S_n$ has pgf
  \begin{align*}
    G_{S_n}(s) = G_1(s)G_2(s)\cdots G_n(s), \quad 0 \leq s \leq 1
  \end{align*}
\end{boks}

\begin{proof}
  Since $X_1, \ldots, X_n$ are independent, the random variables $s^{X_1}, \ldots, s^{X_n}$ are also independent for each $s$ in [0,1], and we get
  \begin{align*}
    G_{S_n}(s) &= E[s^{X_1 + X_2 + \cdots + X_n}]\\
    &= E[s^{X_1}]E[s^{X_2}] \cdots E[s^{X_n}]\\
    &= G_1(s)G_2(s) \cdots G_n(s)
  \end{align*}
  and we are done.
\end{proof}

\newpage

\section{Limit theorems.}
% !TEX root = ../main.tex
When we introduced expected values, we argued that these could be considered averages of a large number of observations. Thus, if we have observations $X_1, X_2, \ldots, X_n$ and we do not know the mean $\mu$, a reasonable approximation ought to be the \textit{sample mean}
\begin{align*}
  \bar{X} = \frac{1}{n} \sum_{k = 1}^{n} X_k
\end{align*}
in other words, the average of $X_1, \ldots, X_n$. Suppose now that the $X_k$ are i.i.d. with mean $\mu$ and variance $\sigma^2$. By the formulas for the mean and variance of sums of independent variables, we get
\begin{align*}
  E[\bar{X}] = E\left[ \frac{1}{n} \sum_{k = 1}^{n} X_k \right] = \sum_{k = 1}^{n} \frac{1}{n} E[X_k] = \mu
\end{align*}
and
\begin{align*}
  Var[\bar{X}] = Var \left[\frac{1}{n} \sum_{k = 1}^{n} X_k\right] = \sum_{k = 1}^{n} \frac{1}{n^2} Var[X_k] = \frac{\sigma^2}{n}
\end{align*}
that is, $\bar{X}$ has the same expected value as each individual $X_k$ and a variance that becomes smaller the larger the value of $n$.

\subsection{The Law of Large Numbers}
ALthough we can nevet guarantee that $|\bar{X} - \mu|$ is smaller than a given $\varepsilon$ we can say that is very likely that $|\bar{X} - \mu|$ is small if $n$ is large. That is the idea behind the following result.
\begin{boks}{Theorem 4.1. (The Law of Large Numbers)}
  Let $X_1, X_2, \ldots$ be a sequence of i.i.d. random variables with mean $\mu$, and let $\bar{X}$ be their sample mean. Then, for every $\varepsilon > 0$
  \begin{align*}
    P(|\bar{X} - \mu| > \varepsilon) \rightarrow 0 \quad
    \text{as} \quad n \rightarrow \infty
  \end{align*}
\end{boks}
\begin{proof}
  Assume that the $X_k$ have finite variance, $\sigma^2 < \infty$.
  Apply Chebyshev's inequality to $\bar{X}$ and let $c = \varepsilon \sqrt{n}/\sigma$.
  Since $E[\bar{X}] = \mu$ and $Var[\bar{X}] = \sigma^2/n$, we get
  \begin{align*}
    P(|\bar{X} - \mu| > \varepsilon) \leq \frac{\sigma^2}{n\varepsilon^2} \rightarrow 0 \quad
    \text{as} \quad n \rightarrow \infty
  \end{align*}
The assumptions of finite variance is neccessary for this proof to work. However, the law of large numbers is tru also if the varinace is infinite, but the proof in that case is more involved and we will not give it.
\end{proof}
We say that $\bar{X}$ \textit{converges in probability} to $\mu$ and write
\begin{align*}
  \bar{X} \xrightarrow{P} \mu \quad \text{as} \quad n \rightarrow \infty
\end{align*}
\begin{boks}{Corollary 4.1}
  Consider an experiment where the event $A$ occurs with probability $p$. Repeat the experiment independently, let $S_n$ be the number of times we get the event $A$ in $n$ trials, and let $f_n = S_n/n$, the relative frequency. Then
  \begin{align*}
    f_n \xrightarrow{P} p \quad \text{as} \quad n \rightarrow \infty
  \end{align*}
\end{boks}
\begin{proof}
  Define the indicators
  \begin{align*}
    I_k =
    \begin{cases}
      1 \qquad \text{if we get $A$ in the $k$th trial} \\
      0 \qquad \text{otherwise}
    \end{cases}
  \end{align*}
for $k = 1, 2, \ldots, n$. Then the $I_k$ are i.i.d. and we know from Section 2.5.1 that they have mean $\mu = p$. Since $f_n$ is the sample mean of the $I_k$, the law of large numbers gives $f_n \xrightarrow{P} p$ as $n \rightarrow \infty$.
\end{proof}
\begin{boks}{Theorem 4.2 (The Central Limit Theorem)}
  Let $X_1, X_2, \ldots$ be i.i.d. random variables with mean $\mu$ and variance $\sigma^2 < \infty$ and let $S_n = \sum_{k = 1}^{n} X_k$. Then, for each $x \in \mathbb{R}$, we have
  \begin{align*}
    \sqrt{n}\frac{\bar{X}_n - \mu}{\sigma} \quad \xrightarrow[n \rightarrow \infty]{d} \quad N(0,1)
  \end{align*}
  as $n \rightarrow \infty$, where $\Phi$ is the cdf of the standard normal distribution.
\end{boks}
\begin{boks}{Definition 4.1}
  Let $X_1, X_2, \ldots$ be a sequence of discrete random variables such that $X_n$ has pmf $p_{X_n}$. If $X$ is a discrete random variable with pmf $p_X$ and
  \begin{align*}
    p_{X_n}(x) \rightarrow p_{X}(x) \quad
    \text{as} \quad n \rightarrow \infty \quad \text{for all} \quad x
  \end{align*}
  then we say that $X_n$ \textit{converges in distribution} to $X$, written $X_n \xrightarrow{d} X$.
\end{boks}
\begin{boks}{Proposition 4.2}
  Let $X_1, X_2, \ldots$ be a sequence of random variables such that $X_n \sim bin(n, p_n)$, where $np_n \rightarrow \lambda > 0$ as $n \rightarrow \infty$, ans let $X \sim \poi(\lambda)$. Then $X_n \xrightarrow{d} X$.
\end{boks}

\subsection{Continuous Limits}
Let us next consider the case when the limiting random variable is continuous. As we already know from the de Moivre-Laplace theorem, the limit can be continuous even if the random variables themselves are not.
\begin{boks}{Definition 4.2}
  Let $X_1, X_2, \ldots$ be a sequence of random variables such that $X_n$ has cdf $F_n$. If $X$ is a continuous random variable with cdf $F$ random
  \begin{align*}
    F_n(x) \rightarrow F(x)
    \quad \text{as} \quad
    n \rightarrow \infty
    \quad \text{for all} \quad
    x \in \mathbb{R}
  \end{align*}
we say that $X_n$ \textit{converges in distribution} to $X$, written $X_n \xrightarrow{d} X$.
\end{boks}
The most important result of this type is the central limit theorem. Another class of important results regarding convergence in distribution deals with the so-called \textit{extreme values}, for example, the minimum and maximum in a sequence of random variables.

\newpage

\section{Markov chains.}
% !TEX root = ../main.tex
Many real-world applications of probability theory have one particular feature that data are collected sequentially in time.
A few examples are weather data, stock market indices, air-pollution data, demographic data, and political tracking polls.
These also have antoher feature in common that successive observations are typically not independent.
We refer to any such collection of abservation as a \textit{stochastic process}.
Formally, a stochastic process is a collection of random variables that take values in a set $S$, the \textit{state space}.
The collection is indexed by another set $T$ the \textit{index set}.
The two most common sets are the natural numbers $T = \{0, 1, 2, \ldots\}$ and the non negative real numbers $T = [0, \infty)$, which usually represent discrete time and continuous time, respectively.
The first index set thus gives a sequence of random variables $\{X_0, X_1, X_2, \ldots \}$ and the second, a  collection of random variables $\{X(t), t \geq 0 \}$, one random variable for each time $t$.
In general, the index set does not have to describe time and is also commenly use to describe spatial location.
The state space can be finite, countably infinite, or uncountable, depending on the application.

\subsection{Discrete-Time Markov Chains}
\begin{boks}{Definition 8.1.}
  Let $X_0, X_1, X_2, \ldots$ be a sequence of discrete random variables, taking values in some set $S$ and that are such that
  \begin{align}\label{eq:markov_egenskaben}
    P(X_{n + 1} = j | X_0 = i_0, \ldots, X_{n - 1} = i_{n - 1}, X_n = i) =
    P(X_{n + 1} = j | X_n = i)
  \end{align}
  for all $i, j, i_0, \ldots, i_{n - 1}$ in $S$ and all $n$. The sequence $\{ X_n \}$ is then called a \textit{Markov chain}. Furthermore \eqref{eq:markov_egenskaben} is called the \textit{Markov property}.
\end{boks}
We ofthen think of the index $n$ as discrete time and say that $X_n$ is the \textit{state} of the chain at time $n$, where the state space $S$ may be finite or coutably infinte.

In general, the probability $P(X_{n + 1} = j | X_n = i)$ depends in $i,j$, and $n$. It is, however, often the case that there is no dependence on $n$. We call such chains \textit{time-homogenous} and restrict our attention to these chains. Since the conditional probability in the definition thus depends only on $i$ and $j$, we use the notation
\begin{align*}
  p_{ij} = P(X_{n + 1} = j | X_n = i), \quad i,j \in S
\end{align*}
and call these the \textit{transition probabilities} of the Markov chain. Thus, if the chain is in state \textit{i}, the probabilities $p_{ij}$ desribe how the chain chooses which state to jump to necxt. Obviously, the transition probabilities have to satisfy the following two criteria:
\begin{align*}
  \text{(a)} \ p_{ij} \geq 0 \quad \text{for all} \ i,j \in S, \qquad
  \text{(b)} \ \sum_{j \in S} p_{ij} = 1 \quad \text{for all} \ i \in S
\end{align*}

\subsubsection{Time Dynamics of a Markov Chain}
The most fundamental aspect of a Markov chain in which we are interested is how it develops over time.
The transition matrix provides us with a despriction od the stepwise behaviour, but suppose that we want to compute the distribution of the chain two steps ahead. Let
\begin{align*}
  p_{ij}^{(2)} = P(X_2 = j | X_0 = i)
\end{align*}
and condition on the intermediate step $X_1$. The law of total probability gives
\begin{align*}
  p_{ij}^{(2)} &= \sum_{k \in S} P(X_2 = j | X_0 = i, X_1 = k)
  P(X_1 = k | X_0 = i) \\
  &=\sum_{k \in S} P(X_2 = j | X_1 = k) P(X_1 = k | X_0 = i) =
  \sum_{k \in S} p_{ik}p_{kj}
\end{align*}
where we used the Markov property for the second-to-last equality. We switched the order between the factors in the sum to get the intuitively appealing last expression; in order to fo from $i$ to $j$ in two steps, we need to visit intermediate step $k$ and jump from there to $j$. Now, recall how matrix multiplication works to help us realize from the expression above that $p_{ij}^{(2)}$ is the $(i, j)$th entry in the matrix $P^2$. Thus, in order to get the two-step transition probabilities, we square the transition matrix. Generally, define the \textit{n-step transition probabilities} as
\begin{align*}
  p_{ij}^{(n)} = P(X_n = j | X_0 = i)
\end{align*}
and let $P^{(n)}$ be the $n$-step transition matrix. Repeating the argument above gives $P^{(n)} = P^n$, the $n$th power of the one-step transition matrix. In particular, this gives the relation
\begin{align*}
  P^{(n + m)} = P^{(n)}P^{(m)}
\end{align*}
for all $m, n$, commonly referred to as the \textit{Chapman-Kolmogorov equations}. Spelld out coordinatewise, they become
\begin{align*}
  p_{ij}^{(n + m)} =  \sum_{k \in S} p_{ik}^{(n)}p_{jk}^{(m)}
\end{align*}
for all $m, n$ and all $i,j \in S$. In words, to go from $i$ to $j$ in $n + m$ steps, we nedd to visit some intermediate step $k$ after $n$ steps. We let $P^{(0)} = I$, the identity matrix.

\subsubsection{Classification of States}
\begin{boks}{Definition 8.2}
  If $p_{ij}^{(n)} > 0$ for some $n$, we say that state $j$ is \textit{accesible} from state $i$, written $i \rightarrow j$. If $j \rightarrow i$, we say that $i$ and $j$ \textit{communicate} and write this $i \leftrightarrow j$.
\end{boks}
In general, if we fix a state $i$ in the state space of a Mrkov vhain, we can find all states that communicate with $i$ and form a \textit{communicating class} containing $i$. It is easy to realize that not only does $i$ communicate with all states in this class but they all communicate with each other. By convention, every state communicates with itself (it can reach itself in $0$ steps) so every state belongs to a class.
If you wish to be more mathemaical, the relastion ``$\leftrightarrow$'' is an equivalence relation and thus divides the state space into equivalence classes that are precisely the communicating classes.
\begin{boks}{Definition 8.3}
  If all states in $S$ communicate with each other, the Markov chain is said to be \textit{irreducible}.
\end{boks}
\begin{boks}{Definition 8.4}
  Consider a state $i \in S$ and the $\tau_i$ be the number of steps it takes for the chain to first visit $i$. Thus
  \begin{align*}
    \tau_i = \min \{ n \geq 1 \ : \ X_n = i \}
  \end{align*}
  where $\tau_i = \infty$ if $i$ is never visited. If $P_i(\tau_i < \infty) = 1$, the state $i$ is said to be \textit{recurrent} and if $P_i(\tau_i < \infty) < 1$, it is said to be \textit{transient}.
\end{boks}
\begin{boks}{Corollary 8.1}
  In an irreducible Markov chain, either all states are trasient or all states are recurrent.
\end{boks}
\begin{boks}{Corollary 8.2}
  Suppose that $S$ is finite. A state $i$ is transient if and only if there is another state $j$ such that $i \rightarrow j$ but $j \nrightarrow i$.
\end{boks}
\begin{boks}{Corollary 8.3}
  If a Markov chain has finite state space, there is at least one recurrent state.
\end{boks}
\begin{boks}{Proposition 8.1}
  State $i$ is
  \begin{align*}
    \text{transient if} \quad &\sum_{n = 1}^\infty p_{ii}^{(n)} < \infty \\
    \text{recurrent if} \quad &\sum_{n = 1}^\infty p_{ii}^{(n)} = \infty
  \end{align*}
\end{boks}

\subsubsection{Stationary Distribution}
\begin{boks}{Definition 8.5}
  Let $P$ be the transition matrix of a Markov chain with state space $S$. A probability distribution $\boldsymbol{\pi} = (\pi_1, \pi_2, \ldots)$ on $S$ satisfying
  \begin{align*}
    \boldsymbol{\pi} P = \boldsymbol{\pi}
  \end{align*}
  is called a \textit{stationary distribution} of the chain.
\end{boks}
\begin{boks}{Proposition 8.2}
  Consider an irreducible Markov chain. If a stationary distribution exists, it is unique.
\end{boks}
\begin{boks}{Proposition 8.3}
  If $S$ is finite and the Markov chain is irreducible, a unique stationary distribution $\boldsymbol{\pi}$ exists.
\end{boks}
\begin{boks}{Definition 8.6}
  Let $i$ be a recurrent state. If $E_i[\tau_i] < \infty$, then $i$ is said to be $positive recurrent$. If $E_i[\tau_i] = \infty$, $i$ is said to be \textit{null recurrent}.
\end{boks}
\begin{boks}{Corollary 8.4}
  For an irreducible Markov chain, there are three possibilities: \textbf{(a)} all states are positive recurrent, \textbf{(b)} all states are null recurrent, and \textbf{(c)} all states are transient.
\end{boks}
\begin{boks}{Proposition 8.4}
  Consider an irreducible Markov chain $\{ X_n \}$. Then,
  \begin{align*}
    \text{A stationary distribution} \ \boldsymbol{\pi} \ \text{exists} \Leftrightarrow \{ X_n \} \ \text{is positive recurrent}
  \end{align*}
  If this is the case, $\boldsymbol{\pi}$ is unique and has $\pi_j > 0$ for all $j \in S$.
\end{boks}

\subsubsection{Convergence to the Stationary Distribution}
\begin{boks}{Definition 8.7}
  Let $p_{ij}^{(n)}$ be the $n$-step transition probabilities of a Markov chain.
  If there exists a probability distribution \textbf{q} on $S$ such that
  \begin{align*}
    p_{ij}^{(n)} \rightarrow q_j \quad \text{as} \quad n \rightarrow \infty \quad \text{for all} i,j \in S
  \end{align*}
  we call \textbf{q} the \textit{limit distribution} of the Markov chain.
\end{boks}
\begin{boks}{Definition 8.8}
  The \textit{period} of state $i$ is defined as
  \begin{align*}
    d(i) = \gcd \{ n \geq 1 \ : \ p_{ii}^{(n)} > 0 \}
  \end{align*}
  the greatest common divisor of lengths of cycles through which it is possible to return to $i$. If $d(i) = 1$, state $i$ is said to be \textit{aperiodic}; otherwise it is called \textit{periodic}. Note that this is a class property, which means if $i \leftrightarrow j$ and either $i$ or $j$ is periodic so is the other. Likewise for aperiodic.
\end{boks}
\begin{boks}{Theorem 8.1}
  Consider an irreducible, positive recurrent, and aperiodic Markov chain with stationary distribution $\boldsymbol{\pi}$ and $n$-step transition probabilities $p_{ij}^{(n)}$. Then
  \begin{align*}
    p_{ij}^{(n)} \rightarrow \pi_j \quad \text{as} \quad n \rightarrow \infty
  \end{align*}
  for all $i,j \in S$.
\end{boks}
An irreducible, positve recurrent, and aperiodic Markov chain is called \textit{ergodic}.
\begin{boks}{Proposition 8.5}
  Consider an ergodic Markov chain with stationary distrbution $\boldsymbol{\pi}$ and choose two states $i$ and $j$. Let $\tau_i$ be the return tome to state $i$, and let $N_j$ be the number of visits to $j$ between consecutive visits to $i$. Then
  \begin{align*}
    E_i[\tau_i] = \frac{1}{\pi_i} \quad \text{and} \quad
    E_i[N_j] = \frac{\pi_j}{\pi_i}
  \end{align*}
\end{boks}
Note that by positive recurrence, all the $E_i[\tau_i]$ are finite and hence all the $\pi_i$ are strictly positive. The $E_i[\tau_i]$ are called the \textit{mean recurrence times}. 

\newpage

\section{Stochastic simulation.}
% !TEX root = ../main.tex
\subsection{Simulation of Continuous Distributions}
\begin{boks}{Proposition 5.2 (The Inverse Transformation Method)}
  Let $F$ be a distribution function that is continuous and strictly increasing. Further, let $U \sim \unif[0,1]$ and define the random variable $Y = F^{-1}(U)$. Then Y has distribution function $F$.
\end{boks}
\begin{proof}
  Start with $F_Y$, the distribution function of $Y$. Take $x$ in the range of $Y$ to obtain
  \begin{align*}
    F_Y(x) &= P(F^{-1}(U) \leq x) \\
    &= P(U \leq F(x)) = F_U(F(x)) = F(x)
  \end{align*}
  where the last equality follows since $F_U(u) = u$ if $0 \leq u \leq 1$. The argument here is $u = F(x)$, which is between $0$ and $1$ since $F$ is a cdf.
\end{proof}

\begin{boks}{Example 5.2}
  Generate an observation from an exponential distribution with parameter $\lambda$.

  Here,
  \begin{align*}
    F(x) = 1- e^{-\lambda x}, \quad x \geq 0
  \end{align*}
  To find the inverse, as usual solve $F(x) = u$, to obtain
  \begin{align*}
    x = F^{-1}(u) = -\frac{1}{\lambda} \log(1 - u), \quad 0 \leq u < 1
  \end{align*}
  Hence if $U \sim \unif[0,1]$, the random variable
  \begin{align*}
  X = -\frac{1}{\lambda} \log(1 - U)
  \end{align*}
  is exp($\lambda$). We can note here that since $1 - U$ is also uniform on $[0,1]$, we might as well take $X = - \log U/\lambda$.
\end{boks}

\begin{boks}{Proposition 5.3 (The Rejection Method)}
  \begin{enumerate}[label = \arabic*.]
    \item Generate $Y$ and $U \sim \unif[0,1]$ independent of each other.
    \item If $U \leq \frac{f(Y)}{cg(Y)}$, set $X = Y$. Otherwise return to step $1$.
  \end{enumerate}
  The random variable $X$ generated by the algorithm has pdf $f$.
\end{boks}
\begin{proof}
  Let us first make sure that the algorithm terminates. The probability in any given step 2 to accept $Y$ is, by Corollary 3.2,
  \begin{align*}
    P((X, Y) \in B) = \int_{\mathbb{R}} P((x, Y) \in B)f_X(x) dx
  \end{align*}
  where
  \begin{align*}
    (Y, U) \quad \text{and} \quad B = \left\{(Y, U) \ : \ U \leq \frac{f(y)}{cg(y)}\right\}
  \end{align*}
  \begin{align*}
    P\left( U \leq \frac{f(y)}{cg(y)} \right) &=
    \int_{\mathbb{R}} P\left( U \leq \frac{f(y)}{cg(y)} \right) g(y)dy \\
    &= \int_{\mathbb{R}} \frac{f(y)}{cg(y)} g(y)dy =
    \frac{1}{c}\int_\mathbb{R}f(y)dy =
    \frac{1}{c}
  \end{align*}
  where we used the independence of $U$ and $Y$ and the fact that $U \sim \unif[0,1]$. Hence, the number of iterations until we accept a value has a geometric distribution with success probability $1/c$. The algorithm therefore always terminates in a number of steps with mean $c$ from which it also follows that we should choose $c$ as small as possible.

  Next we turn to the question of why this gives the corresct distribution. To show this, we will show that the conditional distribution of $Y$, given acceptance, is the same as the distribution of $X$. Recalling the definition of conditional probability and the fact that the probability of acceptance is $1/c$, we get from
  \begin{align*}
    P(A|B) = \frac{P(A \cap B)}{P(B)}
  \end{align*}
  that
  \begin{align*}
    P\left(Y \leq x \ \bigg| \ U \leq \frac{f(y)}{cg(y)} \right) =
    \frac{P\left(Y \leq x \ \cap \ U \leq \frac{f(y)}{cg(y)}\right)}{1/c} = cP\left(Y \leq x \ \cap \ U \leq \frac{f(y)}{cg(y)}\right)
  \end{align*}
  By independence, the joint pdf of $(U, Y)$ is $f(u,y) = g(y)$, and the above expression becomes
  \begin{align*}
    cP\left(Y \leq x \ \cap \ U \leq \frac{f(y)}{cg(y)}\right) &= c \int_{-\infty}^x \int_0^{f(y)/cg(y)} g(y)du \ dy \\
    &= c \int_{-\infty}^x \frac{f(y)}{cg(y)}g(y)dy = P(X \leq x)
  \end{align*}
  which is what we wanted to prove.
\end{proof}

\newpage

\end{document}
